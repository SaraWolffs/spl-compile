\documentclass{report}

\usepackage[a4paper]{geometry}
\usepackage{hyperref}
\usepackage{xcolor}
\usepackage{listings}
\lstdefinelanguage{SPL}{%
	alsoletter={0123456789ABCDEFGHIJKLMNOPQRSTUVWXYZabcdefghijklmnopqrstuvwxyz_+-*/\%=<>!\&|}
	morekeywords={while,else,if,print,return,isEmpty,fst,snd,hd,tl,Int,Char,Bool,Void,True,False,var},%
	sensitive=true,%
	morecomment=[l]{//},%
	morecomment=[n]{/*}{*/},%
	literate=%
		{->}{{$\rightarrow$}}2
		{>=}{{$\geq\:$}}1
		{<=}{{$\leq\:$}}1
		{==}{{$\equiv$}}1
}
\lstset{%
	upquote=true,
	breakatwhitespace=false,
	breaklines=true,
	postbreak=\mbox{\textcolor{gray}{$\hookrightarrow$}\space},
	keepspaces=true,
	basicstyle=\tt\footnotesize,
	commentstyle=\sl,
	keywordstyle=\bf,
	stringstyle=\tt,
	showspaces=false,
	showstringspaces=false,
	showtabs=false,
	tabsize=4,
	basewidth=0.43em,
	columns=[c]fixed,
	texcl=true,
	captionpos=b
}

\author{%
	Sal Wolffs\\
	\small\texttt{s4064542}
}
\date{\today}
\title{SPL-Compile}

\begin{document}

\maketitle%

\tableofcontents%

\chapter{Introduction}
\begin{quote}
	\it
    Prediction is very difficult, especially if it's about the future!
\end{quote}
\section{Language: Rust}
For this project, the Rust language was chosen, which offers a nice balance of
speed, abstraction, safety, functional amenities and imperative control. It's
also a very young language with serious development effort behind it, so it
combines a lot of the niceties language designers explored over the past
decades.

\section{SPL}
SPL is a simple C-like language targeting a didactic stack machine. It borrows
some syntax ideas from the Miranda family (e.g. Haskell), but is imperative at
its core. Those familiar with the C family will easily read it as a limited
subset with a few added features. In this project, we will first implement it,
and then try to extend it into something more expressive.
%\begin{itemize}
%	\item Motivate your language choice
%	\item Introduce spl
%	\item Give some nice examples.
%	\item \ldots
%\end{itemize}

\chapter{Lexing \& Parsing}
\section{Grammar}
To facilitate parsing the language, we made some adjustments to the grammar,
ending up with two versions: one human-friendly ``semantic'' grammar, which
stays close to the specifications given, and one heavily refactored ``parsing''
grammar, which is LL(1) but somewhat hard to recognize as SPL. They are intended
to accept the same language, however, just specified differently for different
purposes.

TODO: rework these notes from the grammar into a pretty list of grammar
adjustments. 
\begin{verbatim}
// Fused FStmt to decrease lookahead (unfuse in semantic analysis)
// Removed negative integer constants, fix in constant folding.
// On top of the below (from grammar.txt), extensive left-factoring, and removal
// of rules so made redundant.

// Op2 = '+' | '-' | '*' | '%'
   | '==' | '<' | '>' | '<=' | '>=' | '!='
   | '&&' | '||'
   | ':' // Replaced by Opl_n, Opr_n, and Op1_n
// ActArgs = Exp [ ',' ActArgs ] // regularized
// Exp = ... // Completely rewritten as Arith_n and Atom
// qualid = id ('.' Selector)* // added, factored out of Stmt, Atom.
// Field = [ Field ( '.' 'hd' | '.' 'tl' | '.' 'fst' | '.' 'snd' ) ] 
// left factored, regularized, factored out Selector, inlined.
// Selector = 'hd' | 'tl' | 'fst' | 'snd' // Added
// Compound = '{' Stmt* '}' // added, factored out of Stmt
// FunType = [ FTypes ] '->' RetType // regularized and inlined
// FTypes = Type [ FTypes ] // replaced by Type* in FunType
// FArgs = (id ',') FArgs // removed left recursion, then regularized.

/***************   ERRATA   ****************/
int = digit+
// greedy lexing would incorrectly lex 5-6 as two ints otherwise. The parser can
// undo this in the expression parser.
\end{verbatim}

In general, there is a tendency to simplify the grammar by accepting more, and
delegating hard problems to later stages.

\section{Lexer}
We did decide to use a separate lexer, which mainly performs a massive
\texttt{match} and then goes into specialized functions for more complicated
tokens (identifiers, integers) and comments. Using a specialized function also
allows dealing with nested comments. Comments are entirely dropped, and strings
are interned as part of lexing, with the store as a public field of the lexer
for later use. Doing interning in the lexer also allows the lexer to do keyword
recognition by preloading keywords into the interner during initialization.

The lexed tokens are made available by making the lexer a Rust Iterator, which
means that repeated calls to \texttt{lexerinstance.next()} yield successive tokens.

Because the parser will want to do 1 token lookahead, we added a
\texttt{.peek()} method to the lexer which gives the next token without
consuming it.

\section{AST\&Parser}
The parser is a manual predictive recursive descent parser with a single token
lookahead. This means that it can only parse an LL(1) grammar, but in return it
never needs to backtrack, yielding linear parse times. The parser's essential
call structure, excluding generic helper functions, is a carbon copy of the
parsing grammar, while the produced AST is a datatype whose structure is almost
a straight copy of the semantic grammar, the difference being the addition of
location annotation and space for adding type information during semantic
analysis.

There is one issue with limiting ourselves to LL(1) grammars, which is that
expressions cannot be unambiguously parsed with correct precedence as an LL(1)
grammar. To get around this, we use a specialized shunting yard parser, which is
pretty much textbook except for the point where it doesn't deal with
parentheses or calls, rather calling back into the main parser to parse those
and then dealing with the result as atoms. The shunting yard uses numerical
precedence, and the number range is stretched to allow future (or
user-specified) insertions, which has the side effect of letting us encode
left/right associativity in the parity of the precedence (so even is left
associative, odd is right), which has the nice side effect of avoiding
equal-precedence differing-fixity operators leaving room for ambiguous parses.

Errors are handled differently according to nature: if the error is a compiler
bug, it panics, terminating compilation with a backtrace to debug the compiler
itself with. If the error is a syntax error in the input, return an error value
to the user with a reason and location. Currently, no errors are accumulated,
but the error-value based design does make this a doable addition.


\chapter{Analyses \& Typing}
\begin{itemize}
	\item New Abstract Syntax Tree? Decorate existing Abstract Syntax Tree?
	\item Error messages?
	\item Polymorphism? Inference? Overloading?
	\item Problems?
	\item\ldots
\end{itemize}

\chapter{Code Generation}
\begin{itemize}
	\item Compilation scheme?
	\item How is data represented? Lists tuples
	\item Semantics style, call-by-reference, call-by-value?
	\item How did you solve overloaded functions?
	\item Polymorphism?
	\item Printing?
	\item Problems?
	\item\ldots
\end{itemize}

\chapter{Extension}
Describe your extension in detail

\chapter{Conclusion}
What does work, what does not etc.

\section{Reflection}
\begin{itemize}
	\item What do you think of the project?
	\item How did it work out?
	\item How did you divide the work?
	\item Pitfalls?
	\item \ldots
\end{itemize}

\appendix
\chapter{Grammar}

AST grammar (subject to minor revision: need to add FStmt to reflect code):
\begin{verbatim}
SPL = Decl+
Decl = VarDecl | FunDecl
VarDecl = ('var' | Type) id '=' Exp ';'
FunDecl = id '(' [ FArgs ] ')' [ '::' FunType ] '{' VarDecl* Stmt+ '}'
RetType = Type | 'Void'
FunType = Type* '->' RetType
Type = BType | '(' Type ',' Type ')' | '[' Type ']' | id
BType = 'Int' | 'Bool' | 'Char'
FArgs = id (',' id)*
Stmt = 'if' '(' Exp ')' Compound [ 'else' Compound ]
     | 'while '(' Exp ')' Compound
     | qualid '=' Exp ';'
     | FunCall ';'
     | 'return' [ Exp ] ';'
Compound = '{' Stmt* '}'
Selector = 'hd' | 'tl' | 'fst' | 'snd'
qualid = id ('.' Selector)*
Exp = Arith_top
Atom = qualid
     | int | char | 'False' | 'True'
     | '(' Exp ')'
     | FunCall
     | '[]'
     | '(' Exp ',' Exp ')'
FunCall = id '(' [ ActArgs ] ')'
ActArgs = Exp ( ',' Exp )*



FORALL n in {1..top}: // MEMO: parse this by shunting yard.
Arith_n = 
        | Arith_{n-1} Opr_n Arith_n
        | Arith_n Opl_n Arith_{n-1}
        | Op1_n Arith_n
        | Arith_{n-1}
Opl_n = Opl WITH prio(Opl) == n
Opr_n = Opr WITH prio(Opr) == n
Op1_n = Op1 WITH prio(Op1) == n
Opl = '+' | '-' | '*' | '%'
    | '==' | '<' | '>' | '<=' | '>=' | '!='
    | '&&' | '||'
Opr = ':'
Op1 = '!' | '-'

Arith_0 = Atom

/***************   TOKENS    ****************/
int = digit+
id = alpha ( '_' | alphanumeric )*
char = "'" any
keys:
'var' 'Void' 'Int' 'Bool' 'Char' 'if' 'else' 'while' 'return' 'hd' 'tl' 'fst'
'snd' 'False' 'True' 
';' '(' ')' '{' '}' ']' ',' '.' 
'&&' '||' '+' '%' 
'<': '<='
'>': '>='
'!': '!='
'=': '=='
'[': '[]'
':': '::'
'-': '->'
'*': '*/'
'//' '/*' // Special case: strip in preprocessing
\end{verbatim}

Parsing grammar:
\begin{verbatim}
SPL = Decl+
Decl = 
    'var' VarInit
  | NonIdType VarInit         {peek in {'Int','Bool','Char','(','['}}
  | id FunOrNamedTypeVarDecl
FunOrNamedTypeVarDecl = 
   FunDef                     {peek in {'('}}
 | VarInit                    {peek in {id} }
VarInit = id '=' Exp ';'
FunDef = Tuplish_id [ '::' FunType ] '{' FStmt* '}'
FStmt = 
    'var' VarInit
  | NonIdType VarInit         {peek in {'Int','Bool','Char','(','['}}
  | KeyWordStmt               {peek in {'if','while','return'}}
  | id StmtOrVarDecl
StmtOrVarDecl = 
   AssignOrCall               {peek in {'.','=','('}}
 | VarInit                    {peek in {id} }
RetType = Typ
FunType = Typ* '->' RetType
Typ  = NonIdType              {peek in {'Int','Bool','Char','(','['}}
     | id
NonIdType = BType | Tuplish_Type | '[' Typ ']'
BType = 'Int' | 'Bool' | 'Char' | 'Void'
Stmt = 'if' '(' Exp ')' Compound [ 'else' Compound ]
 | 'while '(' Exp ')' Compound
 | id AssignOrCall
 | 'return' [ Exp ] ';'
AssignOrCall = 
   Field '=' Exp ';'           {peek in {'.','='}}
 | Tuplish_Exp ';'             {peek in {'('}}
Compound = '{' Stmt* '}'
Selector = 'hd' | 'tl' | 'fst' | 'snd'
Field = ('.' Selector)*
Exp = Arith_top
Atom = id FieldOrCall
 | int | char | 'False' | 'True' | '[]'
 | Tuplish_Exp                       {peek in {'('}}
FieldOrCall = Field             {peek in <otherwise> }
            | Tuplish_Exp       {peek in {'('}}
Tuplish_a = '(' [ a ( ',' a )* ] ')'



FORALL n in {1..top}: // MEMO: parse this by shunting yard.
Arith_n = 
    | Arith_{n-1} Opr_n Arith_n
    | Arith_n Opl_n Arith_{n-1}
    | Op1_n Arith_n
    | Arith_{n-1}
Opl_n = Opl WITH prio(Opl) == n
Opr_n = Opr WITH prio(Opr) == n
Op1_n = Op1 WITH prio(Op1) == n
Opl = '+' | '-' | '*' | '%'
| '==' | '<' | '>' | '<=' | '>=' | '!='
| '&&' | '||'
Opr = ':'
Op1 = '!' | '-'

Arith_0 = Atom

/***************   TOKENS    ****************/
int = digit+
id = alpha ( '_' | alphanumeric )*
char = "'" any
keys:
'var' 'Void' 'Int' 'Bool' 'Char' 'if' 'else' 'while' 'return' 'hd' 'tl' 'fst'
'snd' 'False' 'True' 
';' '(' ')' '{' '}' ']' ',' '.' 
'&&' '||' '+' '%' 
'<': '<='
'>': '>='
'!': '!='
'=': '=='
'[': '[]'
':': '::'
'-': '->'
'*': '*/'
'//' '/*' // Special case: strip in preprocessing
\end{verbatim}
\end{document}
